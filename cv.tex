\documentclass{resume} % Use the custom resume.cls style
\usepackage{cv}
\PassOptionsToPackage{hyphens}{url}\usepackage{hyperref}

%----------------------------------------------------------------------------------------
%	PERSONAL DATA
%----------------------------------------------------------------------------------------\\

\name{Matteo Magnini}
\phone{(+39) 334 38 32 814}
%\skype{\href{skype:matteo.magnini?call}{\texttt{matteo.magnini}}}
\birthday{June 19, 1995} 
\genre{Male}
\country{Italy}
%\address{Via Angeli, 1, 47521 Cesena (FC), Italy}
\address{Cesena (FC), Italy}
%\email{\href{mailto:matteo.magnini@studio.unibo.it}{\texttt{matteo.magnini@studio.unibo.it}}}
\email{\href{mailto:matteo.magnini@unibo.it}{\texttt{matteo.magnini@unibo.it}
}}
\linkedin{\href{https://www.linkedin.com/in/matteo-magnini/}{linkedin.com/in/matteo-magnini}}
\orcid{\href{https://orcid.org/0000-0001-9990-420X}{\texttt{0000-0001-9990-420X}}}
\photo{me.png}

\begin{document}
    
    \begin{rSection}{In short}
        PhD Student in Computer Science and Engineering at the \href{https://disi.unibo.it/it}{Department of Computer Science and Engineering} (DISI), University of Bologna.
        %
        Main research activities comprehends the fields of Artificial Intelligence, Machine Learning, Logic and Neuro-Symbolic techniques.
        %
        
    \end{rSection}
    
    %----------------------------------------------------------------------------------------
    %	EDUCATION SECTION
    %----------------------------------------------------------------------------------------
    
    \begin{rSection}{Education}
        
        \begin{rSubsection}{PhD Student}{\textbf{Nov., 2022 $\rightarrow$ Now}}{Computer Science and Engineering}{University of Bologna, Italy}
            \item Studies in the field of Computer Science and AI, with particular attention to ML, formal logics, neural-symbolic methods, fairness approaches and to the development of new technologies.
        \end{rSubsection}
        
        %\begin{rSubsection}{State Exam}{\textbf{June, 2021}}{Industrial and Information Engineer}{University of Bologna, Italy}
        %    \item Qualification for the profession of engineer.
        %\end{rSubsection}
        
        \begin{rSubsection}{Master's degree}{\textbf{Oct., 2018 $\rightarrow$ Mar., 2021}}{Computer Science and Engineering}{University of Bologna, Italy}
            \item Studies in the field of programming paradigms, computational models, distributed systems, robotics, machine learning, artificial vision, web applications, business intelligence, data mining, big data.
            
            \item 110/110 \emph{cum laude}
        \end{rSubsection}
        \begin{rSubsection}{Master's degree thesis}{\textbf{Mar. 26, 2021}}{An information theory analysis of critical Boolean networks as control software for robots}{\begin{flushright}
                    Relator: Prof. Andrea Roli
            \end{flushright}}
            \item Identifying relationships between robots, controlled by critical random Boolean networks, successfully achieving the assigned task and defined information theory functions on sensors and actuators.
            
            \item External site: \url{https://amslaurea.unibo.it/23062}
        \end{rSubsection}
        % Dissertation: 9/10/2014
        \begin{rSubsection}{Bachelor's degree}{\textbf{Sept., 2015 $\rightarrow$ Dec., 2018}}{Computer Science and Engineering}{University of Bologna, Italy}
            %	{University of Bologna, Seconda Facolt\`a di Ingegneria, Cesena, Italy}%	
            \item Studies in the field of programming languages, operating systems, software engineering, algorithms, networking and web.
            \item 110/110 \emph{cum laude}
        \end{rSubsection}
        
        \begin{rSubsection}{Bachelor's degree thesis}{\textbf{Dec. 14, 2018}}{Ottimizzazione Combinatorica mediante Deep Reinforcement Learning: Sperimentazione nella Logistica di Magazzino.}{\begin{flushright}
                    Relator: Prof. Gianluca Moro
            \end{flushright}}
            \item Training and analysis of a neural network using Deep Reinforcement Learning to optimise the allocation of commodity in a warehouse with respect to picking frequency.
            
            \item External site: \url{https://amslaurea.unibo.it/17000}
        \end{rSubsection}
        
        \begin{rSubsection}{High-school diploma}{\textbf{2009 $\rightarrow$ 2014}}{Scientific curriculum}{Liceo Scientifico ``A. Righi'', Cesena (FC), Italy}
            \item Final mark 100/100
        \end{rSubsection}
        
    \end{rSection}
    
    %----------------------------------------------------------------------------------------
    %	WORK EXPERIENCE SECTION
    %----------------------------------------------------------------------------------------
    
    \begin{rSection}{Experience}
        
        \begin{rSubsection}{Visiting Researcher}{\textbf{Mar., 2024 $\rightarrow$ Jun., 2024}}{Department of Informatics}{University of Oslo, Norway}
            \item Supervisor: Prof. \href{mailto:anaoz@ifi.uio.no}{Ana Ozaki}
        \end{rSubsection}

        \begin{rSubsection}{Teaching assistant for the course ``Distributed Systems''}{\textbf{Sept., 2023 $\rightarrow$ Now}}{School of Engineering and Architecture (24 hours)}{University of Bologna, Italy}
            \item Asynchronous programming, distributed principles and architectures, consensus algorithms, ReSTfull web-services, containers, agent-based technologies and middlewares.
            \item Supervisor: Prof. \href{mailto:andrea.omicini@unibo.it}{Andrea Omicini}
            \item Course Info: \url{https://apice.unibo.it/xwiki/bin/view/Courses/Ds2324}
        \end{rSubsection}
        
        \begin{rSubsection}{Teacher at professional education course}{\textbf{May 5, 2023}}{BPER -- Data Analytics (8 hours)}{\href{https://www.bbs.unibo.eu/}{BBS}}
            \item Talks topics: eXplainable Artificial Intelligence (XAI): A Gentle Introduction.
            \item References:  Prof. \href{mailto:andrea.omicini@unibo.it}{Andrea Omicini}
        \end{rSubsection}
        
        \begin{rSubsection}{Teaching assistant for the course ``Fondamenti di Informatica''}{\textbf{Feb., 2023 $\rightarrow$ June, 2023}}{School of Engineering and Architecture (35 hours)}{University of Bologna, Italy}
            \item Computer architecture, representation of information, algorithms and data structures, C programming language.
            \item Supervisor: Dr. \href{mailto:roberto.casadei@unibo.it}{Roberto Casadei}
            \item Course Info: \url{https://apice.unibo.it/xwiki/bin/view/Courses/Finf2223/}
        \end{rSubsection}
        
        \begin{rSubsection}{Teaching assistant for the course ``Sistemi Distribuiti''}{\textbf{Sept., 2022 $\rightarrow$ Jan. 2023}}{School of Engineering and Architecture (24 hours)}{University of Bologna, Italy}
            \item Asynchronous programming, distributed principles and architectures, consensus algorithms, ReSTfull web-services, containers, agent-based technologies and middlewares.
            \item Supervisor: Prof. \href{mailto:andrea.omicini@unibo.it}{Andrea Omicini}
            \item Course Info: \url{https://apice.unibo.it/xwiki/bin/view/Courses/Sd2223}
        \end{rSubsection}
        
        \begin{rSubsection}{Teacher at professional education course}{\textbf{May 17-19, 2022}}{IFTS course (6 hours)}{\href{http://www.formart.it/home}{FORMart}}
            \item Talks topics: Business Intelligence and Big Data.
            \item References:  Prof. \href{mailto:a.ricci@unibo.it}{Alessandro Ricci}
        \end{rSubsection}
        
        \begin{rSubsection}{Teaching assistant for the course ``Fondamenti di Informatica''}{\textbf{Feb., 2022 $\rightarrow$ June, 2022}}{School of Engineering and Architecture (35 hours)}{University of Bologna, Italy}
            \item Computer architecture, representation of information, algorithms and data structures, C programming language.
            \item Supervisor: Dr. \href{mailto:roberto.casadei@unibo.it}{Roberto Casadei}
            \item Course Info: \url{https://apice.unibo.it/xwiki/bin/view/Courses/FINF2022}
        \end{rSubsection}
        
        \begin{rSubsection}{Research Fellow}{\textbf{Oct., 2021 $\rightarrow$ Oct., 2022}}{ \href{https://disi.unibo.it/it}{Department of Computer Science and Engineering} (DISI)}{University of Bologna, Italy}
            \item Project title: ``Strumenti di logica computazionale per estrazione e iniezione di conoscenza simbolica da e verso predittori subsimbolici''
            \item Goal: producing a software for the symbolic extraction/injection from/into sub-symbolic predictors
            \item Supervisor: Prof. \href{mailto:andrea.omicini@unibo.it}{Andrea Omicini}
        \end{rSubsection}
        
        \begin{rSubsection}{ETL product specialist}{\textbf{May 2021 $\rightarrow$ Oct., 2021}}{Healthcare division}{Onit Group, Cesena, Italy}
            \item Datawarehouse, database, web-services, customer care.
            %\item Reference: \href{mailto:vbuda@onit.it}{Vladimiro Buda}
            \item Reference: \href{mailto:czanella@onit.it}{Cecilia Zanella}
        \end{rSubsection}
        
        \begin{rSubsection}{Tutor DM}{\textbf{Feb., 2021 $\rightarrow$ Mar. 2021}}{Educational services sector}{University of Bologna, Italy}
            \item Technical support for mixed teaching and lessons supervision for Covid-19 prevention.
        \end{rSubsection}
        
        \begin{rSubsection}{Master's Degree students Representative in Course Council}{\textbf{Jul., 2019 $\rightarrow$ Mar. 2021}}{Computer Science and Engineering}{University of Bologna, Italy}
            \item \url{https://corsi.unibo.it/magistrale/IngegneriaScienzeInformatiche/coordinatore-consiglio}
        \end{rSubsection}
        
        \begin{rSubsection}{Master's Degree students Representative in AQ Council}{\textbf{Jul., 2019 $\rightarrow$ Mar. 2021}}{Computer Science and Engineering}{University of Bologna, Italy}
            \item \url{https://corsi.unibo.it/magistrale/IngegneriaScienzeInformatiche/commissioni}
        \end{rSubsection}
        
        \begin{rSubsection}{Internship}{\textbf{Oct., 2017 $\rightarrow$ Jen., 2018}}{LIAM Lab}{Spilamberto (MO), Italy}
            \item Study of OPCUA communication protocol
            \item Supervisor: Prof. \href{mailto:matteo.sartini@unibo.it}{Matteo Sartini} 
        \end{rSubsection}
        
    \end{rSection}
    
    %----------------------------------------------------------------------------------------
    %	SOFTWARE DEVELOPMENT
    %----------------------------------------------------------------------------------------
    %\newpage
    \begin{rSection}{Development of Research-related Software}
        
        \begin{rSubsection}{PSyKE}{\textbf{Nov., 2021 $\rightarrow$ Ongoing}}{}{}
            \item a (python) platform for symbolic knowledge extraction from sub-symbolic predictors.
            \item \url{https://github.com/psykei/psyke-python}
        \end{rSubsection}
    
    \begin{rSubsection}{PSyKI}{\textbf{Apr., 2022 $\rightarrow$ Ongoing}}{}{}
        \item a (python) platform for symbolic knowledge injection into sub-symbolic predictors.
        \item \url{https://github.com/psykei/psyki-python}
    \end{rSubsection}
        
    \end{rSection}
    
    %----------------------------------------------------------------------------------------
    %	LANGUAGE SECTION
    %----------------------------------------------------------------------------------------
    
    \begin{rSection}{Language self-assessment}
        \begin{center}
            \begin{tabular}{|c|c|c|c|c|c|}
                \hline
                &\textbf{Listening}&\textbf{Reading}&\textbf{Interaction}&\textbf{Speaking}&\textbf{Writing}\\\hline
                \textbf{Italian}&\multicolumn{5}{c}{Native language}\vline\\\hline
                \textbf{English}&C1&C1&C1&C1&C1\\\hline
                \textbf{French}&B2&B2&B1&B1&B1\\\hline
                \textbf{Norwegian}&A2&A2&A1&A1&A1\\\hline
            \end{tabular}
        \end{center}
    \end{rSection}
    
    % \pagebreak
    
    %----------------------------------------------------------------------------------------
    %	TECHNICAL STRENGTHS SECTION
    %----------------------------------------------------------------------------------------
    % \newpage
    \begin{rSection}{Technical Strengths}
        \begin{tabular}{ @{} >{\bfseries}l @{\hspace{6ex}} l }
            Programming Paradigms	& imperative, object oriented, functional, logic\\
            Software configuration 	& Windows and Linux installation and configuration\\
            Programming Languages 	& Python, Java, Scala, C, C++, C\#, Prolog, JavaScript\\
            Data Analysis Tools		& R, Pandas, NumPy, Scikitlearn, Tensorflow, Pytorch\\
            Networking 				& Socket (TCP \& UDP), HTTP, RESTful WebAPI \\
            Databases 				& SQL, MySQL, MongoDB, Vertica \\
            Development tools 		& Git, Maven, Gradle, Docker, Continuous Integration \\
            Markup languages 		& Markdown, \LaTeX, HTML\\
            IDEs 					& PyCharm, IntelliJ Idea, Visual Studio, Eclipse
        \end{tabular}						
    \end{rSection}
    
    
    %----------------------------------------------------------------------------------------
    %	ADDITIONAL INFO SECTION
    %----------------------------------------------------------------------------------------
    
    \begin{rSection}{Additional information}
        
        \item \textbf{About me}: I am an engineer and software developer with experience in machine learning and data mining. I can both work in team or alone. I always prefer deep analysis and discussion on the problem along with wide research in literature before solving it.
        
        \item \textbf{Interests}: Artificial Intelligence, Machine Learning, Logic, Data Mining, Artificial Vision, Robotics (swarm robotics and adaptation).
        
    \end{rSection}
    
    %----------------------------------------------------------------------------------------
    %	SIGNATURE SECTION
    %----------------------------------------------------------------------------------------
    \vspace{2em} %4em
    \begin{flushright}
        \today
        \\
        \vspace{1em}
        \Large$\mathcalligra{Matteo}$  $\mathcalligra{Magnini}$
        
        %\includegraphics[width=6cm]{./firma.pdf}
    \end{flushright}

    \pagebreak
    \MakeUppercase{\bf List of Publications} % Section title
    \sectionlineskip
    \hrule % Horizontal line
    \vspace{-1cm}
    \nocite{*}
    %\bibliographystyle{chronological}
    \bibliographystyle{elsarticle-num} 
    \renewcommand\refname{}
    \bibliography{papers}
    
\end{document}