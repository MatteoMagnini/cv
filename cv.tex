\documentclass{resume} % Use the custom resume.cls style
\usepackage{cv}
\PassOptionsToPackage{hyphens}{url}
\usepackage{hyperref}
\addbibresource{bibliography.bib}

%----------------------------------------------------------------------------------------
%	PERSONAL DATA
%----------------------------------------------------------------------------------------\\

\name{Matteo Magnini}
\phone{(+39) 334 38 32 814}
\birthday{June 19, 1995} 
\genre{Male}
\country{Italy}
\town{47521, Cesena (FC)}
\address{via dell'Università 50}
%\email{\href{mailto:matteo.magnini@studio.unibo.it}{\texttt{matteo.magnini@studio.unibo.it}}}
\email{\href{mailto:matteo.magnini@unibo.it}{\texttt{matteo.magnini@unibo.it}}}
\github{\href{https://github.com/MatteoMagnini}{github.com/MatteoMagnini}}
\linkedin{\href{https://www.linkedin.com/in/matteo-magnini/}{linkedin.com/in/matteo-magnini}}
\orcid{\href{https://orcid.org/0000-0001-9990-420X}{\texttt{0000-0001-9990-420X}}}
\googlescholar{\href{https://scholar.google.com/citations?user=iJDPEaUAAAAJ\&hl=en}{\texttt{iJDPEaUAAAAJ}}}
\photo{me.png}

\begin{document}
    
    \begin{rSection}{In short}
        PhD Student in Computer Science and Engineering at the \href{https://disi.unibo.it/it}{Department of Computer Science and Engineering} (DISI), University of Bologna.
        %
        Main research activities comprehends the fields of Artificial Intelligence, Machine Learning, Logic and Neuro-Symbolic techniques.
        %
        
    \end{rSection}
    
    %----------------------------------------------------------------------------------------
    %	EDUCATION AND CARREER SECTION
    %----------------------------------------------------------------------------------------
    \begin{rSection}{Education and Carreer}

        %----------------------------------------------------------------------------------------
        %	EDUCATION SECTION
        %----------------------------------------------------------------------------------------
        
        \begin{rSubsection2}{Education}

            \item\textbf{ Ph.D. Student }\hfill \textbf{Nov., 2022 $\rightarrow$ Now}
            \\Computer Science and Engineering, University of Bologna, Italy.
            \\Studies in the field of Computer Science and AI, with particular attention to ML, formal logics, neural-symbolic methods, fairness approaches and the development of new technologies.
            \\Supervisor: Prof. \href{mailto:andrea.omicini@unibo.it}{Andrea Omicini}
            %
            \item\textbf{ Master's degree }\hfill \textbf{Oct., 2018 $\rightarrow$ Mar., 2021}
            \\Computer Science and Engineering, University of Bologna, Italy.
            \\Studies in the field of programming paradigms, computational models, distributed systems, robotics, machine learning, artificial vision, web applications, business intelligence, data mining, big data.
            \\110/110 \emph{cum laude}
            %
            \item\textbf{ Master's degree thesis }\hfill \textbf{Mar., 26, 2021}
            \\An information theory analysis of critical Boolean networks as control software for robots
            \\Relator: Prof. \href{mailto:andrea.roli@unibo.it}{Andrea Roli}
            \\Identifying relationships between robots, controlled by critical random Boolean networks, successfully achieving the assigned task and defined information theory functions on sensors and actuators.
            \\External site: \url{https://amslaurea.unibo.it/23062}
            %
            \item\textbf{ Bachelor's degree }\hfill \textbf{Sept., 2015 $\rightarrow$ Dec., 2018}
            \\Computer Science and Engineering, University of Bologna, Italy.
            \\Studies in the field of programming languages, operating systems, software engineering, algorithms, networking and web.
            \\110/110 \emph{cum laude}
            %
            \item\textbf{ Bachelor's degree thesis }\hfill \textbf{Dec., 14, 2018}
            \\Ottimizzazione Combinatorica mediante Deep Reinforcement Learning: Sperimentazione nella Logistica di Magazzino.
            \\Relator: Prof. \href{mailto:gianluca.moro@unibo.it}{Gianluca Moro}
            \\Training and analysis of a neural network using Deep Reinforcement Learning to optimise the allocation of commodity in a warehouse with respect to picking frequency.
            \\External site: \url{https://amslaurea.unibo.it/17000}
            %
            \item\textbf{ High-school diploma }\hfill \textbf{2009 $\rightarrow$ 2014}
            \\Scientific curriculum, Liceo Scientifico ``A. Righi'', Cesena (FC), Italy.
            \\Final mark 100/100
        \end{rSubsection2}

        %----------------------------------------------------------------------------------------
        %	CERTIFICATIONS SECTION
        %----------------------------------------------------------------------------------------

        \begin{rSubsection2}{Certifications}

            \item\textbf{ State Exam }\hfill \textbf{June, 2021}
            \\Industrial and Information Engineer, University of Bologna, Italy.
            \\Qualification for the profession of engineer.

        \end{rSubsection2}

        %----------------------------------------------------------------------------------------
        %	RESEACH CONTRACTS SECTION
        %----------------------------------------------------------------------------------------

        \begin{rSubsection2}{Research Contracts}

            \item\textbf{ Ph.D. Student }\hfill \textbf{Nov., 2022 $\rightarrow$ Now}
            \\Computer Science and Engineering, University of Bologna, Italy.
            \\Ph.D. course covered by the victory of a grant from the Italian Ministry of Research.
            Initially worked within the \href{https://expectation.ehealth.hevs.ch/posts/home/}{Expectation} project, designing and developing neuro-symbolic methods and tools.
            Then, I worked within the \href{https://www.aequitas-project.eu/}{Aequitas} project, focusing on fairness in AI and ML.
            \\Supervisor: Prof. \href{mailto:andrea.omicini@unibo.it}{Andrea Omicini}
            %
            \item\textbf{ Research Fellow }\hfill \textbf{Oct., 2021 $\rightarrow$ Oct., 2022}
            \\Department of Computer Science and Engineering (DISI), University of Bologna, Italy.
            \\Project title: ``Strumenti di logica computazionale per estrazione e iniezione di conoscenza simbolica da e verso predittori subsimbolici''.
            Goal: producing a software for the symbolic extraction/injection from/into sub-symbolic predictors.
            \\Supervisor: Prof. \href{mailto:andrea.omicini@unibo.it}{Andrea Omicini}

        \end{rSubsection2}

        %----------------------------------------------------------------------------------------
        %	OTHER CONTRACTS SECTION
        %----------------------------------------------------------------------------------------

        \begin{rSubsection2}{Other Contracts}

            \item\textbf{ ETL product specialist }\hfill \textbf{May, 2021 $\rightarrow$ Oct., 2021}
            \\Healthcare division, Onit Group, Cesena, Italy.
            \\Data-warehouse, database, web-services, customer care.
            \\Reference: \href{mailto:czanella@onit.it}{Cecilia Zanella}
            %
            \item\textbf{ Internship }\hfill \textbf{Oct., 2017 $\rightarrow$ Jan., 2018}
            \\LIAM Lab, Spilamberto (MO), Italy.
            \\Study of OPCUA communication protocol.
            \\Supervisor: Prof. \href{mailto:matteo.sartini@unibo.it}{Matteo Sartini}

        \end{rSubsection2}


    \end{rSection}

    %\newpage

    %----------------------------------------------------------------------------------------
    %	PUBLICATIONS SECTION
    %----------------------------------------------------------------------------------------

    \begin{rSection}{Scientific Publications}

        %----------------------------------------------------------------------------------------
        %	BIBLIOMETRICS SECTION
        %----------------------------------------------------------------------------------------

        \begin{rNoListSubsection}{Bibliometrics}{}{}{}
            \input{scholar}
        \end{rNoListSubsection}

        %----------------------------------------------------------------------------------------
        %	JOURNAL PUBLICATIONS SECTION
        %----------------------------------------------------------------------------------------

        % Adjust spacing between bibliography items
        \setlength\bibitemsep{.5\baselineskip} % Adjust the value to change spacing

        \begin{rNoListSubsection}{Journal publications sorted by time}{}{}{}
            \nocite{*}
            \DeclareFieldFormat{labelnumberwidth}{#1}
            \printbibliography[type=article,heading=none,omitnumbers=true]{}
        \end{rNoListSubsection}

        %----------------------------------------------------------------------------------------
        %	CONFERENCE PUBLICATIONS SECTION
        %----------------------------------------------------------------------------------------

        \begin{rNoListSubsection}{Contributions in conference proceedings sorted by time}{}{}{}
            \nocite{*}
            \DeclareFieldFormat{labelnumberwidth}{#1}
            \defbibfilter{conferences}{
              type=inproceedings or
              type=incollection
            }
            \printbibliography[filter=conferences,heading=none,omitnumbers=true]{}
        \end{rNoListSubsection}

        %----------------------------------------------------------------------------------------
        %	OTHER PUBLICATIONS (BOOKS, PROCEEDINGS CHAPTERS, ETC.) SORTED BY TIME SECTION
        %----------------------------------------------------------------------------------------

%        \begin{rNoListSubsection}{Other publications sorted by time}{}{}{}
%            \nocite{*}
%            \DeclareFieldFormat{labelnumberwidth}{#1}
%            \defbibfilter{other}{
%              not type=inproceedings and
%              not type=incollection and
%              not type=article
%            }
%            \printbibliography[filter=other,heading=none,omitnumbers=true]{}
%        \end{rNoListSubsection}

    \end{rSection}

    %\newpage

    %----------------------------------------------------------------------------------------
    %	SCIENTIFIC ACTIVITIES SECTION
    %----------------------------------------------------------------------------------------

    \begin{rSection}{Scientific Activities}

        %----------------------------------------------------------------------------------------
        %	PARTICIPATION IN RESEARCH GROUPS SECTION
        %----------------------------------------------------------------------------------------

        \begin{rSubsection2}{Participation in Research Groups}
            \item\textbf{ PSLab Research group }\hfill \textbf{Oct., 2021 $\rightarrow$ Now}
            \\This is the primary research group I worked with since the conclusion of my two years master.
            \\The \href{https://pslab-unibo.github.io/}{Pervasive Software Lab} is at the forefront of research in pervasive and distributed computing, dedicated to shaping the future of intelligent, adaptive, and scalable systems.
        \end{rSubsection2}

        %----------------------------------------------------------------------------------------
        %	ORGANIZATION OF INTERNATIONAL CONFERENCES SECTION
        %----------------------------------------------------------------------------------------

        \begin{rSubsection2}{Organization of International Conferences}
            %
            \item \href{https://dl2024.w.uib.no/organization/}{the 37th International Workshop on Description Logics (DL 2024)}
            \\Local organization team member.
            %
        \end{rSubsection2}

        %----------------------------------------------------------------------------------------
        %	PARTICIPATION IN PC OF INTERNATIONAL CONFERENCES SECTION
        %----------------------------------------------------------------------------------------

        \begin{rSubsection2}{Participation in Program Committees\\ of International Conferences}
            %
            \item \href{https://extraamas.ehealth.hevs.ch/archive.html#organizations-2022}{the 4th International Workshop on EXplainable and TRAnsparent AI and Multi-Agent Systems (EXTRAAMAS 2022)}
            %
            \item \href{https://apice.unibo.it/xwiki/bin/view/Event/Aaai2023}{the 37th Annual AAAI Conference on Artificial Intelligence (AAAI 2023)}
            %
            \item \href{https://apice.unibo.it/xwiki/bin/view/Event/Prima2023}{the 5th International Workshop on EXplainable and TRAnsparent AI and Multi-Agent Systems (EXTRAAMAS 2023)}
            %
            \item \href{https://web.archive.org/web/20240225110652/https://www.stai.uk/stai-23-iclp}{the Safe and Trustworthy AI Workshop (STAI 2023)}
            %
            \item \href{https://fairnesscluster.github.io/aimmes23.github.io/index.html}{the 1st Workshop on AI bias: Measurements, Mitigation, Explanation Strategies (AIMMES 2024)}
            %
            \item \href{https://apice.unibo.it/xwiki/bin/view/Event/Aaai2024}{the 38th Annual AAAI Conference on Artificial Intelligence (AAAI 2024)}
            %
            \item \href{https://aequitas-aod.github.io/aequitas-ecai24.github.io/pc-member.html}{Second AEQUITAS on Fairness and Bias in AI (AEQUITAS 2024)}
            %
            \item \href{https://aaai.org/conference/aaai/aaai-25/}{the 39th Annual AAAI Conference on Artificial Intelligence (AAAI 2025)}
            %
            \item \href{https://extraamas.ehealth.hevs.ch/index.html}{the 7th International Workshop on EXplainable and TRAnsparent AI and Multi-Agent Systems (EXTRAAMAS 2025)}
            %
            \item \href{https://ansya-workshop.github.io/2025}{the 1st International Workshop on Advanced Neuro-Symbolic Applications (ANSYA 2025)}
        \end{rSubsection2}

        %----------------------------------------------------------------------------------------
        %	REVIEWING FOR INTERNATIONAL JOURNALS SECTION
        %----------------------------------------------------------------------------------------

        \begin{rSubsection2}{Reviewing for International Journals}
            \item \href{https://link.springer.com/journal/10458}{Autonomous Agents and Multi-Agent Systems (ISSN: 1387-2532)}, 2023
            %
            \item \href{https://link.springer.com/journal/13042}{International Journal of Machine Learning and Cybernetics (ISSN: 1868-8071)}, 2024
            %
            \item \href{https://www.jair.org/index.php/jair/index}{Journal of Artificial Intelligence Research (ISSN: 1076-9757)}, 2024-2025
            %
            \item \href{https://link.springer.com/journal/10742}{Health Services and Outcomes Research Methodology (ISSN: 1387-3741)}, 2025
            %
            \item \href{https://link.springer.com/journal/10916}{Journal of Medical Systems (ISSN: 1573-689X)}, 2025
            %
        \end{rSubsection2}

        %----------------------------------------------------------------------------------------
        %	TALKS IN INTERNATIONAL CONFERENCES SECTION
        %----------------------------------------------------------------------------------------

        \begin{rSubsection2}{Talks in International Conferences}
            \item 37th Italian Conference on Computational Logic (CILC 2022)
            \\\href{https://apice.unibo.it/xwiki/bin/view/Talk/KinsCilc2022}{KINS: Knowledge Injection via Network Structuring}
            %
            \item 24th International Conference on Principles and Practice of Multi-Agent Systems (PRIMA 2022)
            \\\href{https://apice.unibo.it/xwiki/bin/view/Talk/PsykitutorialPrima2022}{Symbolic Knowledge Injection via PSyKI. A Tutorial} (material preparation)
            \\\href{https://apice.unibo.it/xwiki/bin/view/Talk/PsykiPrima2022}{Symbolic Knowledge Extraction via PSyKE. A Tutorial} (material preparation)
            %
            \item 21st International Conference of the Italian Association for Artificial Intelligence (AIXIA 2022)
            \\\href{https://apice.unibo.it/xwiki/bin/view/Talk/CtlAixia2022}{Bridging Symbolic and Sub-Symbolic AI: Towards Cooperative Transfer Learning in Multi-Agent Systems}
            %
            \item Doctoral Consortium of the 20th International Conference on Principles of Knowledge Representation and Reasoning (KR 2023)
            \\\href{https://apice.unibo.it/xwiki/bin/view/Talk/SymbolicTransferLearning}{Symbolic Transfer Learning through Knowledge Manipulation Methods}
        \end{rSubsection2}

        \begin{rSubsection2}{Other Talks}
            \item 4rd International Workshop on EXplainable and TRAnsparent AI and Multi-Agent Systems (EXTRAAMAS 2022)
            \\\href{https://apice.unibo.it/xwiki/bin/view/Talk/PsykiExtraamas2022}{On the Design of PSyKI: a Platform for Symbolic Knowledge Injection into Sub-Symbolic Predictors}
            %
            \item 23rd Workshop ``From Objects to Agents'' (WOA 2022)
            \\\href{https://apice.unibo.it/xwiki/bin/view/Talk/KillWoa2022}{A view to a KILL: Knowledge Injection via Lambda Layer}
            %
            \item Advanced School in Artificial intelligence in Emilia Romagna 2023 (ASAI 2023)
            \\\href{https://apice.unibo.it/xwiki/bin/view/Talk/XaiAsaiErBertinoro2023}{eXplainable Artificial Intelligence (XAI): A Gentle Introduction}
            %
            \item 1st Workshop on AI bias: Measurements, Mitigation, Explanation Strategies (AIMMES 2024)
            \\\href{https://apice.unibo.it/xwiki/bin/view/Talk/IntersectionalityAimmes2024}{Mitigating Intersectional Fairness: a Practical Approach with FaUCI}
            %
            \item 37th International Workshop on Description Logics (DL 2024)
            \\\href{https://dl2024.w.uib.no/overview/}{Active Learning Ontologies from LLMS: first results}
            %
            \item Second AEQUITAS Workshop on Fairness and Bias in AI | co-located with ECAI 2024 (AEQUITAS 2024)
            \\\href{https://apice.unibo.it/xwiki/bin/view/Talk/FauciAequitas2024}{Enforcing Fairness via Constraint Injection with FaUCI}
        \end{rSubsection2}


        \begin{rSubsection2}{Awards}
            \item \textbf{Best Paper Award}
            \\
            \textit{4th International Workshop on Telemedicine and E-Health Evolution in the New Era of Social Distancing (TELMED 2025)} co-located with the \textit{23rd IEEE International Conference on Pervasive Computing and Communications (PerCom 2025)}, Washington, USA.
            \\
            ``Neuro-symbolic AI for Supporting Chronic Disease Diagnosis and Monitoring''

        \end{rSubsection2}

        %----------------------------------------------------------------------------------------
        %	INTERNATIONAL EXPERIENCE SECTION
        %----------------------------------------------------------------------------------------

        \begin{rSubsection2}{International Experience}

            \item\textbf{ Visiting Researcher }\hfill \textbf{Mar., 2024 $\rightarrow$ Jun., 2024}
            \\Department of Informatics, University of Oslo, Norway
            \\Design and development of the exact learner framework by using LLMs as teacher.
            \\Supervisor: Prof. \href{mailto:anaoz@ifi.uio.no}{Ana Ozaki}

        \end{rSubsection2}

    \end{rSection}

    %\newpage

    %----------------------------------------------------------------------------------------
    %	TEACHING SECTION
    %----------------------------------------------------------------------------------------

    \begin{rSection}{Teaching}

        %----------------------------------------------------------------------------------------
        %	TUTORING SECTION
        %----------------------------------------------------------------------------------------

        \begin{rSubsection2}{Tutoring}
            \item\textbf{ Teaching assistant for the course ``Fondamenti di Informatica'' }\hfill \textbf{Feb., 2025 $\rightarrow$ June, 2025}
            \\School of Engineering and Architecture (35 hours), University of Bologna, Italy
            \\Computer architecture, representation of information, algorithms and data structures, C programming language.
            \\Supervisor: Dr. \href{mailto:roberto.casadei@unibo.it}{Roberto Casadei}
            \\Course Info: \url{https://virtuale.unibo.it/course/view.php?id=63744}
            %
            \item\textbf{Teaching assistant for the course ``Distributed Systems'' }\hfill \textbf{Sept., 2024 $\rightarrow$ Now}
            \\School of Engineering and Architecture (24 hours), University of Bologna, Italy
            \\Asynchronous programming, distributed principles and architectures, consensus algorithms, ReSTfull web-services, containers, agent-based technologies and middlewares.
            \\Supervisor: Prof. \href{mailto:andrea.omicini@unibo.it}{Andrea Omicini}
            \\Course Info: \url{https://apice.unibo.it/xwiki/bin/view/Courses/Ds2425}
            %
            \item\textbf{ Teaching assistant for the course ``Distributed Systems'' }\hfill \textbf{Sept., 2023 $\rightarrow$ Jan., 2024}
            \\School of Engineering and Architecture (24 hours), University of Bologna, Italy
            \\Asynchronous programming, distributed principles and architectures, consensus algorithms, ReSTfull web-services, containers, agent-based technologies and middlewares.
            \\Supervisor: Prof. \href{mailto:andrea.omicini@unibo.it}{Andrea Omicini}
            \\Course Info: \url{https://apice.unibo.it/xwiki/bin/view/Courses/Ds2324}
            %
            \item\textbf{ Teaching assistant for the course ``Fondamenti di Informatica'' }\hfill \textbf{Feb., 2023 $\rightarrow$ June, 2023}
            \\School of Engineering and Architecture (35 hours), University of Bologna, Italy
            \\Computer architecture, representation of information, algorithms and data structures, C programming language.
            \\Supervisor: Dr. \href{mailto:roberto.casadei@unibo.it}{Roberto Casadei}
            \\Course Info: \url{https://virtuale.unibo.it/course/view.php?id=42157}
            %
            \item\textbf{ Teaching assistant for the course ``Sistemi Distribuiti'' }\hfill \textbf{Sept., 2022 $\rightarrow$ Jan., 2023}
            \\School of Engineering and Architecture (24 hours), University of Bologna, Italy
            \\Asynchronous programming, distributed principles and architectures, consensus algorithms, ReSTfull web-services, containers, agent-based technologies and middlewares.
            \\Supervisor: Prof. \href{mailto:andrea.omicini@unibo.it}{Andrea Omicini}
            \\Course Info: \url{https://apice.unibo.it/xwiki/bin/view/Courses/Sd2223}
            %
            \item\textbf{ Teaching assistant for the course ``Fondamenti di Informatica'' }\hfill \textbf{Feb., 2022 $\rightarrow$ June, 2022}
            \\School of Engineering and Architecture (35 hours), University of Bologna, Italy
            \\Computer architecture, representation of information, algorithms and data structures, C programming language.
            \\Supervisor: Dr. \href{mailto:roberto.casadei@unibo.it}{Roberto Casadei}
            \\Course Info: \url{https://virtuale.unibo.it/course/view.php?id=34867}
            %
        \end{rSubsection2}

        %----------------------------------------------------------------------------------------
        %	EXTRA-INSTITUTIONAL TEACHING SECTION
        %----------------------------------------------------------------------------------------

        \begin{rSubsection2}{Extra-institutional Teaching}
            %
            \item\textbf{ Teacher at professional education course }\hfill \textbf{May 5, 2023},
            \\BPER -- Data Analytics (8 hours), \href{https://www.bbs.unibo.eu/}{BBS}
            \\Talks topics: eXplainable Artificial Intelligence (XAI): A Gentle Introduction.
            \\References:  Prof. \href{mailto:andrea.omicini@unibo.it}{Andrea Omicini}
            %
            \item\textbf{ Teacher at professional education course }\hfill \textbf{May 17-19, 2022}
            \\IFTS course (6 hours), \href{http://www.formart.it/home}{FORMart}
            \\Talks topics: Business Intelligence and Big Data.
            \\References:  Prof. \href{mailto:a.ricci@unibo.it}{Alessandro Ricci}
            %
        \end{rSubsection2}

        %----------------------------------------------------------------------------------------
        %	SUPERVISION OF GRADUATE STUDENTS SECTION
        %----------------------------------------------------------------------------------------

        \begin{rSubsection2}{Supervision of Graduate Students}

            \item Riccardo Squarcialupi, Master's Degree, Department of Computer Science and Engineering, University of Bologna, Italy
            \\ Actively Ontology Learning from Large Language Models, 2024 (\href{https://amslaurea.unibo.it/id/eprint/32247/}{Amslaurea}).
            %
            \item Michelangelo Ungolo, Bachelor's Degree, School of Pure and Applied Sciences, University of Urbino, Italy
            \\ Progettazione e sviluppo di un chatbot basato su LLM a supporto del paziente iperteso, 2024 (\href{https://apice.unibo.it/xwiki/bin/view/Thesis/ChatbotLLMIpertensioneUngolo2024}{Apice}).
            %
            \item Giulia Costa, Bachelor's Degree, School of Pure and Applied Sciences, University of Urbino, Italy
            \\ Intelligenza artificiale e salute: addestramento di un LLM per il supporto del paziente iperteso, 2024 (\href{https://apice.unibo.it/xwiki/bin/view/Thesis/AddestramentoLLMPerPazienteIpertesoCosta2024}{Apice}).

        \end{rSubsection2}

    \end{rSection}

    %\newpage

    %----------------------------------------------------------------------------------------
    %	OTHER ACTIVITIES AND SKILLS SECTION
    %----------------------------------------------------------------------------------------

    \begin{rSection}{Other Activities and Skills}

        %----------------------------------------------------------------------------------------
        %	FACULTY ACTIVITY SECTION
        %----------------------------------------------------------------------------------------

        \begin{rSubsection2}{Faculty Activity}
            %
            \item\textbf{ Tutor DM }\hfill \textbf{Feb., 2021 $\rightarrow$ Mar. 2021}
            \\Educational services sector, University of Bologna, Italy
            \\Technical support for mixed teaching and lessons supervision for Covid-19 prevention.
            %
            \item\textbf{ Master's Degree students Representative in Course Council }\hfill \textbf{Jul., 2019 $\rightarrow$ Mar., 2021}
            \\Computer Science and Engineering, University of Bologna, Italy
            \\\url{https://corsi.unibo.it/magistrale/IngegneriaScienzeInformatiche/coordinatore-consiglio}
            %
            \item\textbf{ Master's Degree students Representative in AQ Council }\hfill \textbf{Jul., 2019 $\rightarrow$ Mar., 2021}
            \\Computer Science and Engineering, University of Bologna, Italy
            \\\url{https://corsi.unibo.it/magistrale/IngegneriaScienzeInformatiche/commissioni}
            %
        \end{rSubsection2}

        %----------------------------------------------------------------------------------------
        %	SOFTWARE DEVELOPMENT
        %----------------------------------------------------------------------------------------

        \begin{rSubsection2}{Lead Designer of Research-related Software}
            %
            \item\textbf{ PSyKE }\hfill \textbf{Nov., 2021 $\rightarrow$ Ongoing}
            \\Python platform for symbolic knowledge extraction from sub-symbolic predictors.
            \\\url{https://github.com/psykei/psyke-python}
            %
            \item\textbf{ PSyKI }\hfill \textbf{Apr., 2022 $\rightarrow$ Ongoing}
            \\Python platform for symbolic knowledge injection into sub-symbolic predictors.
            \\\url{https://github.com/psykei/psyki-python}
            %
            \item\textbf{ Hyper-tension bot }\hfill \textbf{Jan., 2024 $\rightarrow$ Ongoing}
            \\Telegram hyper-tension bot for monitoring patients' blood pressure and providing general help.
            \\\url{https://github.com/MatteoMagnini/hyperTensionBot}
        \end{rSubsection2}

        %----------------------------------------------------------------------------------------
        %	LANGUAGE SECTION
        %----------------------------------------------------------------------------------------

        \begin{rNoListSubsection}{Language self-assessment}{}{}{}
            \begin{center}
                \begin{tabular}{|c|c|c|c|c|c|}
                    \hline
                    &\textbf{Listening}&\textbf{Reading}&\textbf{Interaction}&\textbf{Speaking}&\textbf{Writing}\\\hline
                    \textbf{Italian}&\multicolumn{5}{c}{Native language}\vline\\\hline
                    \textbf{English}&C1&C1&C1&C1&C1\\\hline
                    \textbf{French}&B2&B2&B1&B1&B1\\\hline
                    \textbf{Norwegian}&A1&A1&A1&A1&A1\\\hline
                \end{tabular}
            \end{center}
        \end{rNoListSubsection}

        %----------------------------------------------------------------------------------------
        %	TECHNICAL STRENGTHS SECTION
        %----------------------------------------------------------------------------------------
        % \newpage
        \begin{rNoListSubsection}{Technical Strengths}{}{}{}
            \begin{tabular}{ @{} >{\bfseries}l @{\hspace{6ex}} l }
                Programming Paradigms	& imperative, object oriented, functional, logic\\
                Software configuration 	& Windows and Linux installation and configuration\\
                Programming Languages 	& Python, Java, Scala, C, C++, C\#, Prolog, JavaScript\\
                Data Analysis Tools		& R, Pandas, NumPy, Scikitlearn, Tensorflow, Pytorch\\
                Networking 				& Socket (TCP \& UDP), HTTP, RESTful WebAPI \\
                Databases 				& SQL, MySQL, MongoDB, Vertica \\
                Development tools 		& Git, Maven, Gradle, Docker, Continuous Integration \\
                Markup languages 		& Markdown, \LaTeX, HTML\\
                IDEs 					& PyCharm, IntelliJ Idea, Visual Studio, Eclipse
            \end{tabular}
        \end{rNoListSubsection}

        %----------------------------------------------------------------------------------------
        %	ADDITIONAL INFO SECTION
        %----------------------------------------------------------------------------------------

        %\begin{rSubsection2}{Additional information}
            %
            %\item \textbf{About me}: I am an engineer and software developer with experience in machine learning and data mining. I can both work in team or alone. I always prefer deep analysis and discussion on the problem along with wide research in literature before solving it.
            %
            %\item \textbf{Interests}: Artificial Intelligence, Machine Learning, Logic, Data Mining, Artificial Vision, Robotics (swarm robotics and adaptation).
            %
        %\end{rSubsection2}

    \end{rSection}
    
    %----------------------------------------------------------------------------------------
    %	SIGNATURE SECTION
    %----------------------------------------------------------------------------------------
    \vspace{2em} %4em
    \begin{flushright}
        \today
        \\
        \vspace{1em}
        \Large$\mathcalligra{Matteo}$  $\mathcalligra{Magnini}$
        
        %\includegraphics[width=6cm]{./firma.pdf}
    \end{flushright}
    
\end{document}